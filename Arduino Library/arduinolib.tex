% !TEX TS-program = pdflatex
% !TEX encoding = UTF-8 Unicode

\documentclass[11pt]{article} % use larger type; default would be 10pt

\usepackage[utf8]{inputenc} % set input encoding (not needed with XeLaTeX)


%%% PAGE DIMENSIONS
\usepackage{geometry} % to change the page dimensions
\geometry{margin=1in} % for example, change the margins to 2 inches all round

\usepackage{graphicx} % support the \includegraphics command and options

% \usepackage[parfill]{parskip} % Activate to begin paragraphs with an empty line rather than an indent

%%% PACKAGES
\usepackage[default]{cantarell}
\usepackage[T1]{fontenc}
\usepackage[none]{hyphenat}
\usepackage{booktabs} % for much better looking tables
\usepackage{array} % for better arrays (eg matrices) in maths
\usepackage{paralist} % very flexible & customisable lists (eg. enumerate/itemize, etc.)
\usepackage{verbatim} % adds environment for commenting out blocks of text & for better verbatim
\usepackage{subfig} % make it possible to include more than one captioned figure/table in a single float

%%% HEADERS & FOOTERS
\usepackage{fancyhdr} % This should be set AFTER setting up the page geometry
\pagestyle{fancy} % options: empty , plain , fancy
\lhead{RobotOpen Arduino Library Reference\\Version 1.0.x Release\\www.RobotOpen.biz}
\rhead{\includegraphics[height=42pt]{robotopen.png}}
\lfoot{}\cfoot{\thepage}\rfoot{}
\setlength\headsep{30pt}
\setlength\footskip{0pt}

\begin{document}

\section*{}
\section{Introduction}

\subsection{Overview}
The RobotOpen Arduino Library provides an easy to use abstraction layer to quickly begin programming and controlling your robots with RobotOpen. The 1.0.x release is currently compatible with the RobotOpen Control Shield for Arduino coupled with the boards listed below. 

\subsection{Supported Hardware}

\begin{itemize}
\item Ethernet Arduino
\item Arduino Uno + Ethernet Shield
\item Arduino Duemilanove + Ethernet Shield
\item Sparkfun Ethernet Pro
\end{itemize}

\subsection{Recommended Hardware}
\begin{itemize}
\item Windows Based PC with Java 6+
\item Logitech Dual Action USB Game Pad
\item Ethernet Arduino + RobotOpen Control Shield
\end{itemize}

\subsection{Installation}
Before installing the RobotOpen library, first verify that you are running the latest version of the Arduino IDE. The latest version of the RobotOpen Arduino library can be found at www.robotopen.biz. Download and extract the zip file. You should be left with a folder named "RobotOpen". You can then place this folder into your Arduino installation directory. On Windows this will be located under Arduino/libraries. On OS X you will need to right click the Arduino.app file and select "Show Package Contents". Navigate to Contents/Resources/Java/libraries and place the RobotOpen folder in this directory. If you are upgrading your version of the RobotOpen library you can safely delete the "RobotOpen" directory before copying in the new version.

\newpage

\subsection{Driver Station App}
The Driver Station app is a simple Java application to allow control of your RobotOpen powered robot. This can be obtained from www.robotopen.biz. Please view the getting started video listed on the site to learn more.

\subsection{Examples}
Now that you have the RobotOpen library loaded up, simply open the Arduino IDE and select File >> Examples >> RobotOpen >> TankDrive. This is the simplest example bunded into the RobotOpen library that shows how you should model your RobotOpen projects. For this example, all of the code is provided to get a basic robot up and running in a Tank Drive configuration. By default, the Logitech Dual Action USB Game Pad will work out of the box. If you will be using a Mac to drive the robot you will need to change LOGITECH\_LEFTY to LOGITECH\_LEFTY\_MAC and LOGITECH\_RIGHTY to LOGITECH\_RIGHTY\_MAC. This is due to the fact that OS X maps USB gamepads differently.

\newpage

\section{RobotOpen Class}

The RobotOpen class is the core of the RobotOpen library. It is important to note that the inner workings of RobotOpen use what are called "bundles" to send and receive data between the driver station and robot controller. Every bundle is identified by a bundle ID that is a single character. Typically 0-9, a-z, or A-Z. When accessing the low level data in RobotOpen, you will use these bundle IDs.

\subsection{RobotOpen::begin()}
\textbf{Returns: void}\\\\
This function initializes communications and configures RobotOpen. This should only be called once from the setup() function in the Arduino environment.
\subsection{RobotOpen::pollDS()}
\textbf{Returns: void}\\\\
This function processes any data that has been received from the driver station and updates all joystick objects. It should be called at the beginning of every program loop.
\subsection{RobotOpen::outgoingDS()}
\textbf{Returns: void}\\\\
This function notifies RobotOpen that all of the robot code has finished executing and any data waiting to be sent back to the driver station should happen now. This should be called at the end of every program loop.
\subsection{RobotOpen::setPWM(int pwmChannel, int value)}
\textbf{int pwmChannel}: The PWM output to set. See the Digital Sidecar IO Constants near the end of this document for possible values. Example: SIDECAR\_PWM7.\\
\textbf{int value}: The PWM value to set. Use a joystick makePWM function here or manually pass a value between 0-255.\\
\textbf{Returns: void}\\\\
This function sets a motor or PWM value on a specific PWM output of the digital sidecar. 
\subsection{RobotOpen::enabled()}
\textbf{Returns: boolean}\\\\
This will return to you a boolean of the current state of the robot. True corresponds to enabled. False corresponds to disabled.

\subsection{RobotOpen::publishAnalog(int pin, unsigned char bundleID)}
\textbf{int pin}: The analog pin on the RobotOpen Shield that you would like to send back to the DS. The Analog Input Constants listed below can be passed in for this value.\\
\textbf{unsigned char bundleID}: A unique single character that identifies this bundle. This should be passed in single quotes. Typical values include 0-9, a-z, and A-Z.\\
\textbf{Returns: void}\\\\
This method measures the voltage of the specified analog pin and sends it back to the driver station as a value between 0-1023. The bundleID is simply a method to identify the data "bundle" on your driver station.
\subsection{RobotOpen::publishDigital(int pin, unsigned char bundleID)}
\textbf{int pin}: The digital pin on the RobotOpen Shield that you would like to send back to the DS. The Digital Sidecar IO Constants listed below can be passed in for this value.\\
\textbf{unsigned char bundleID}: A unique single character that identifies this bundle. This should be passed in single quotes. Typical values include 0-9, a-z, and A-Z.\\
\textbf{Returns: void}\\\\
This method measures a digital sidecar pin state (0vdc or 5vdc) and sends it back to the driver station as either a 0 (0vdc signal) or 255 (5vdc signal). The bundleID is simply a method to identify the data "bundle" on your driver station.
\subsection{RobotOpen::publishByte(unsigned char value, unsigned char bundleID)}
\textbf{unsigned char value}: The single byte (8 bit) value that you would like to send back to the DS.\\
\textbf{unsigned char bundleID}: A unique single character that identifies this bundle. This should be passed in single quotes. Typical values include 0-9, a-z, and A-Z.\\
\textbf{Returns: void}\\\\
This method sends a single byte back to the driver station. The bundleID is simply a method to identify the data "bundle" on your driver station.
\subsection{RobotOpen::publishInt(unsigned int value, unsigned char bundleID)}
\textbf{unsigned int value}: The int (16 bit) value that you would like to send back to the DS.\\
\textbf{unsigned char bundleID}: A unique single character that identifies this bundle. This should be passed in single quotes. Typical values include 0-9, a-z, and A-Z.\\
\textbf{Returns: void}\\\\
This method sends a single int back to the driver station. The bundleID is simply a method to identify the data "bundle" on your driver station.
\newpage
\subsection{RobotOpen::publishLong(long value, unsigned char bundleID)}
\textbf{long value}: The long (32 bit) value that you would like to send back to the DS.\\
\textbf{unsigned char bundleID}: A unique single character that identifies this bundle. This should be passed in single quotes. Typical values include 0-9, a-z, and A-Z.\\
\textbf{Returns: void}\\\\
This method sends a single long back to the driver station. The bundleID is simply a method to identify the data "bundle" on your driver station.
\subsection{RobotOpen::getBundleSize(unsigned char bundleID)}
\textbf{Returns: int}\\\\
All data from the driver station is sent in bundles. Normally you will only deal with joystick values coming from your driver station. Each joystick should be in its own bundle. Joystick 1 is typically in bundle '0' and Joystick 2 is typically in bundle '1'. This function allows you to get the size (number of bytes) of a particular bundle that you are receiving from the DS. If the bundle cannot be found, -1 is returned.
\subsection{RobotOpen::getComponent(unsigned char bundleID, int componentIndex)}
\textbf{Returns: int}\\\\
All data from the driver station is sent in bundles. Normally you will only deal with joystick values coming from your driver station. Each joystick should be in its own bundle. Joystick 1 is typically in bundle '0' and Joystick 2 is typically in bundle '1'. This function allows you to get the value from a specific joystick component or byte in your joystick bundle. componentIndex allows you to specify a byte index between 0 (first component) and the number of components in the bundle - 1. Valid values range between 0-255. An error will be returned as -1.
\newpage
\section{USBJoystick Class}

The USBJoystick class allows creation of usb objects from bundle IDs to act as an abstraction layer.\\\\ \textbf{Example}: USBJoystick usb1('0');\\
This creates a USBJoystick object that can now be accessed as usb1.function() using the values in bundle '0'.

\subsection{USBJoystick(unsigned char bundleID)}
The constructor for creating a new USBJoystick object. bundleID specifies the bundle to create the USBJoystick object from. Joystick 1 is typically in bundle '0' and Joystick 2 is typically in bundle '1'.
\subsection{joystick.getIndex(int index)}
\textbf{int index}: The component index between 0 (first component) and the number of components in the bundle - 1.\\
\textbf{Returns: int}\\\\
This function allows you to grab the raw value of any component in this USBJoystick. Valid values range between 0-255. An error will be returned as -1.
\subsection{joystick.makePWM(int index, char mode)}
\textbf{int index}: The joystick axis to create the PWM value from. See Logitech Gamepad Constants below. A raw joystick index can also be passed in here.\\
\textbf{char mode}: How to generate the PWM value (INVERT or NORMAL)\\
\textbf{Returns: int}\\\\
This function will return a PWM value between 0-255 if the joystick axis/component you pass is valid or 127 (neutral) if there is an error. You will normally use this function chained with a setPWM function such as:\\
RobotOpen::setPWM(SIDECAR\_PWM9, usb1.makePWM(LOGITECH\_LEFTY, NORMAL));
\subsection{joystick.getBtn(int index, char mode)}
\textbf{int index}: The joystick button to read. See Logitech Gamepad Constants below. A raw joystick index can also be passed in here.\\
\textbf{char mode}: How to generate the returned boolean (INVERT or NORMAL)\\
\textbf{Returns: boolean}\\\\
This function will return true if the specified button is being pressed or false if it is not being pressed (if the mode is NORMAL). The opposite holds true if the mode is INVERT.
\subsection{joystick.getDpad(int index, unsigned char position, char mode)}
\textbf{int index}: The D-Pad component. See Logitech Gamepad Constants below. This will typically be LOGITECH\_DPAD or LOGITECH\_DPAD\_MAC\\
\textbf{unsigned char position}: The position of the D-Pad to check against (e.g. UP\_RIGHT).\\
\textbf{char mode}: How to generate the returned boolean (INVERT or NORMAL)\\
\textbf{Returns: boolean}\\\\
This function will return true if the specified d-pad direction is being pressed or false if it is not being pressed (if the mode is NORMAL). The opposite holds true if the mode is INVERT.

\newpage
\section{Constants}

\subsection{Analog Inputs}
\begin{itemize}
\item ANALOG0 - 0x00
\item ANALOG1 - 0x01
\item ANALOG2 - 0x02
\item ANALOG3 - 0x03
\item ANALOG4 - 0x04
\item ANALOG5 - 0x05
\end{itemize}

\subsection{Digital Sidecar IO}

\begin{itemize}
\item SIDECAR\_DIGITAL1 - 0x07
\item SIDECAR\_DIGITAL2 - 0x06
\item SIDECAR\_DIGITAL3 - 0x05
\item SIDECAR\_DIGITAL4 - 0x04
\item SIDECAR\_DIGITAL5 - 0x03
\item SIDECAR\_DIGITAL6 - 0x02
\item SIDECAR\_DIGITAL7 - 0x09
\item SIDECAR\_DIGITAL8 - 0x08
\item SIDECAR\_PWM1 - 0x01
\item SIDECAR\_PWM2 - 0x02
\item SIDECAR\_PWM3 - 0x03
\item SIDECAR\_PWM4 - 0x04
\item SIDECAR\_PWM5 - 0x05
\item SIDECAR\_PWM6 - 0x06
\item SIDECAR\_PWM7 - 0x07
\item SIDECAR\_PWM8 - 0x08
\item SIDECAR\_PWM9 - 0x09
\item SIDECAR\_PWM10 - 0x0a
\end{itemize}

\newpage

\subsection{Logitech Gamepad on Windows DS}

\begin{itemize}
\item LOGITECH\_RIGHTY - 0x00
\item LOGITECH\_RIGHTX - 0x01
\item LOGITECH\_LEFTY - 0x02
\item LOGITECH\_LEFTX - 0x03
\item LOGITECH\_DPAD - 0x04
\item LOGITECH\_BTN1 - 0x05
\item LOGITECH\_BTN2 - 0x06
\item LOGITECH\_BTN3 - 0x07
\item LOGITECH\_BTN4 - 0x08
\item LOGITECH\_BTN5 - 0x09
\item LOGITECH\_BTN6 - 0x0a
\item LOGITECH\_BTN7 - 0x0b
\item LOGITECH\_BTN8 - 0x0c
\item LOGITECH\_BTN9 - 0x0d
\item LOGITECH\_BTN10 - 0x0e
\item LOGITECH\_LEFTBTN - 0x0f
\item LOGITECH\_RIGHTBTN - 0x10
\end{itemize}

\newpage

\subsection{Xbox 360 Controller on Windows DS}

\begin{itemize}
\item XBOX\_LEFTY - 0x00
\item XBOX\_LEFYX - 0x01
\item XBOX\_RIGHTY - 0x02
\item XBOX\_RIGHTX - 0x03
\item XBOX\_TRIGGERS - 0x04 (NOTE: This is Analog)
\item XBOX\_A\_BTN - 0x05
\item XBOX\_B\_BTN - 0x06
\item XBOX\_X\_BTN - 0x07
\item XBOX\_Y\_BTN - 0x08
\item XBOX\_LSHOULDER\_BTN - 0x09
\item XBOX\_RSHOULDER\_BTN - 0x0a
\item XBOX\_BACK\_BTN - 0x0b
\item XBOX\_START\_BTN - 0x0c
\item XBOX\_LEFTBTN - 0x0d
\item XBOX\_RIGHTBTN - 0x0e
\item XBOX\_DPAD - 0x0f
\end{itemize}

\newpage

\subsection{Logitech Gamepad on OS X DS}

\begin{itemize}
\item LOGITECH\_RIGHTY\_MAC - 0x0f
\item LOGITECH\_RIGHTX\_MAC - 0x0e
\item LOGITECH\_LEFTY\_MAC - 0x0d
\item LOGITECH\_LEFTX\_MAC - 0x0c
\item LOGITECH\_DPAD\_MAC - 0x10
\item LOGITECH\_BTN1\_MAC - 0x00
\item LOGITECH\_BTN2\_MAC - 0x01
\item LOGITECH\_BTN3\_MAC - 0x02
\item LOGITECH\_BTN4\_MAC - 0x03
\item LOGITECH\_BTN5\_MAC - 0x04
\item LOGITECH\_BTN6\_MAC - 0x05
\item LOGITECH\_BTN7\_MAC - 0x06
\item LOGITECH\_BTN8\_MAC - 0x07
\item LOGITECH\_BTN9\_MAC - 0x08
\item LOGITECH\_BTN10\_MAC - 0x09
\item LOGITECH\_LEFTBTN\_MAC - 0x0a
\item LOGITECH\_RIGHTBTN\_MAC - 0x0b
\end{itemize}

\newpage

\subsection{D-Pad Positions}

\begin{itemize}
\item UP - 0x3F
\item UP\_LEFT -0x1F
\item UP\_RIGHT - 0x5F
\item DOWN - 0xBF
\item DOWN\_LEFT - 0xDF
\item DOWN\_RIGHT - 0x9F
\item LEFT - 0xFF
\item RIGHT - 0x7F
\end{itemize}

\subsection{Joystick Modifiers}

\begin{itemize}
\item NORMAL - 0x00
\item INVERT - 0x01
\end{itemize}

\end{document}
